\documentclass[10pt,twocolumn]{article} 

% required packages for Oxy Comps style
\usepackage{oxycomps} % the main oxycomps style file
\usepackage{times} % use Times as the default font
\usepackage[style=numeric,sorting=nyt]{biblatex} % format the bibliography nicely

\usepackage{amsfonts} % provides many math symbols/fonts
\usepackage{listings} % provides the lstlisting environment
\usepackage{amssymb} % provides many math symbols/fonts
\usepackage{graphicx} % allows insertion of grpahics
\usepackage{hyperref} % creates links within the page and to URLs
\usepackage{url} % formats URLs properly
\usepackage{verbatim} % provides the comment environment
\usepackage{xpatch} % used to patch \textcite

\addbibresource{references.bib}
\DeclareNameAlias{default}{last-first}

\xpatchbibmacro{textcite}
  {\printnames{labelname}}
  {\printnames{labelname} (\printfield{year})}
  {}
  {}

\pdfinfo{
    /Title (Computational Queries: An Interactive Art Installation)
    /Author (Joaquín Madrid Larrañaga)
}

\title{Computational Queries: An Interactive Art Installation - Ethical Considerations}

\author{Joaquín Madrid Larrañaga}
\affiliation{Occidental College}
\email{jmadridlarra@oxy.edu}

\begin{document}

\maketitle

\section{Introduction}\label{sec:intro}

Since the 1960s, interactive art exhibits have ethralled artsts and computer scientists alike\cite{trifonova_software_2008}.  By creating digital projections, responsive animations, and live video feeds, artists have blurred the lines between traditional art and technology.  Many of these installations incorporate visuals and audio and ask participants to move, speak, or interact with physical pieces of the exhibit.  As interesting as these exhibits sound, there are many ethical concerns relating to physical accomodations for people with disabilities and content matter of the interactive art form. Specifically, this paper discusses the many ethical concerns relating to ``Computational Queries,'' an interactive art Installation by Joaquín Madrid Larrañaga. 

\section{Project Overview}\label{sec:overview}

``Computational Queries'' is an interactive art installation that incorporates user movement and responsive projections to create a geometric pattern that shifts and changes according to the position of the user.  A camera interprets the position of a user's hand using open source hand tracking technology and the xy coordinates of the user's hand is fed into the program which generates an immediate response in the animation at that coordinate. In this way, a user can wave their hand around in front of the camera and the geometric patten will flow and shift in real time. Similar to a game, this installation also has additional levels that users can achieve by completing certain tasks.  For example, if a user waves their hand over a certain coordinate, a different color may appear.  If the user places both of their hands at that coordinate and moves them in opposite directions, a space will be opened in the pattern and a new pattern will emerge.  This new level will have a different type of interaction and a different task that must be completed to move on to the next level.  After 5 levels, the installation will return back to the beginning.  As such, users will be able to spend as little or as much time interacting with the installation as they would like. Users who spend more time will discover more easter eggs, but users who spend less time will still be able to interact with and enjoy the installation without feeling confused about how to operate the installation.  In order to aid with any confusion, there will be visual hints and motions to encourage user interaction. 

\section{Ethical Design}\label{sec:design}

In the textbook ``Introduction to Art: Design, Context, and Meaning,'' \cite{blood_introduction_nodate}the authors caution against different types of unethical practices when creating an art piece.  Chapter 11, ``Art and Ethics'' warns against `appropriation' or the act of incorporating another artist's work into their own without any change or edits and without acknowledgement or credit. Similar to plagarism, this type of artistic appropriation not only steals intellectual property from the original artist, but also damages the credibility of the emerging artist.  While artists often borrow ideas, aesthetics, or even entire pieces from other artists, ``Introduction to Art...'' makes the distinction that these artists are intentionally altering the original idea in order to comment on it's original meaning or to recontextualize it in a new setting. In this way, new artists are still bringing their own intentions and ideas to the original work.  However, it is difficult to characterize exactly what `recontextualize' means. 

``Computational Queries'' draws a lot of inspiration from Akash Nihilani, an artist known for his 3D geometric illusions. Specifically, his interactive geometric series called ``Projections'' from 2015.  These interactive projections incorporate different geometric patterns that flow and ebb as users run their hands along the wall that the projections are screened on. As these geometric patterns move, bright colors are revealed in the space between each shape.  In a very similar way, ``Computational Queries'' will ask users to wave their hands in front of geometric patterns in order to reveal another `level' of the game.  While ``Computational Queries'' will have a different color scheme and type of user interaction than ``Projections,'' the two installations are still very similar.  

It is worth mentioning that Chapter 11 also mentions ethical considerations relating to politics, religion, and sexuality.  However, ``Computational Queries'' does not include any displays of politics, religion, or sexuality so no further discussion into these subjects will be needed.

\section{Ethical Interaction}\label{sec:interaction}

At the highest level, this installation focuses on human movement as the primary way of interacting with a computer.  Unfortunately, not all users will be able to interact with the installation.  Hand amputees will not be able to interact since the hand tracking software will not be able to accurately capture their arm movements. Furthermore, users with motor inhibitions may not be able to accurately complete the motions needed to advance to the next level.  As it stands currently, ``Computational Queries'' is not suited for users without traditional motor capabilities. 

In addition, this installation relies heavily on visual feedback for the user.  Users are visually instructed how to interact with the exhibit and then, when they are interacting, the installation responds to their movements visually.  Furthernore, people who interact with the exhibit will be visually prompted hints on screen as well as given visual feedback as a metric for success through the levels. As such, users with visual impairments will not experience any difference as they physically interact with the installation. 

\section{Ethical Computation}\label{sec:computation}
This installation makes use of several open source softwares. However it is imporant to mention the difference between art appropriation as defined above and ethical software incorporation.  The open source software movement aims to encourage the creation of software that is widely shared and available free of charge \cite{jackson_how_nodate}. ``Computational Queries'' collects user input through a camera that transmits a live video feed into the computer. This video feed is then analyzed using open source hand tracking software called handtrack.js \cite{handTrack.js_Dibia2017}.  While it is easy to cite the software in the context of this paper, this literature review of interactive art installations prior to 2008 suggests that there is not a precedent for citing open source software that is used in an art installation.  As such, the considerations that apply to art appropriation are typically not followed for software when discussing art installtions. 

\section{Suggestions for Improvment}\label{sec:suggestions}
After detailing all of the ethical shortcomings listed above, below are some concrete suggestions to improve the ethical considerations of this art installation. 

\subsection{Proper Credits}\label{subsec:credits}
As mentioned in \S 3, proper credit where credit is due is integral from an academic standpoint.  Proper credits to Akash Nihilani as the inspiration to this piece must be included in the installation if not in the projection itself. This may take form as a plaque, a pamphlet, or an overlay on the projection.  Along these lines, any other designers or projection art that influence ``Computational Queries'' should be included. 

\subsection{Accessibility}\label{subsec:accessibility}
\S 4 details the many accessibility issues related to interacting with the art installation. Providing additional ways to interact with the piece becomes integral when dealing with differently abled people. Providing a joystick controller, handheld mouse, or eye tracking software as inputs to determine the position of the movement allows differently abled people alternatives ways to interact that don't involve their hands. 

In addition, steps must be taken to include blind/low vision participants. By including a compelling soundscape that changes as a user interacts, the installation can mimic the visual outputs aurally. For example, as the user interacts with the pattern by moving their hands, the music could change or add a sound.  In this way, blind and low vision users can still interact with the exhibit. 

\subsection{Ethically Compliant Software}\label{subsec:software}
``Computational Queries'' does not collect nor store data from people who use the installation. However, since this project uses open source softwares, it is important to confirm that these softwares also do not collect nor store user data. Along the same lines, special considerations should be made to ensure that these open sourced softwares employ ethically sourced data sets. 

\section{Conclusion}\label{sec:conclusion}
This paper aims to critically examine the interactive art installation, ``Computational Queries''. Through many ethical lenses, this project falls short of being equitable and fails to correctly situate itself among other installations.  This exhibit does not fairly credit all contributing creatives, essentially ostracizes differently abled people and potentially makes use of unethical open source softwares. Though several suggestions for improvement were given, the project as it stands does not meet the lowest of ethical standards and therefore should not continue without making critical changes to address the severe ethical concerns raised. 

\printbibliography 

\end{document}
